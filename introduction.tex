\section{Introduction}
\label{sec:introduction}

Many security applications have a need to inspect the internal workings
of software. Host-based intrusion systems, malware analyses, and
digital forensics all depend to some degree on being able to obtain
information about software that is by design undocumented and hidden
from public view. Thus, to operate correctly, security software is
typically built on \emph{reverse engineering}, the art and practice of
elucidating the undocumented principles on which software is built.

Unfortunately, reverse engineering is expensive, time consuming, and
requires a high degree of expertise. The problem is exacerbated by the
fact that, to protect against tampering, security applications are often
hosted in environments separated from the target being inspected, such
as a separate virtual machine. Because of this, their visibility into
the target is often limited to low-level features such as memory and CPU
state, and any higher-level information must be reconstructed based on
reverse engineered knowledge.

This problem, which we will refer to as the \emph{introspection
problem}, has been approached by a number of recent research efforts
such as Virtuoso~\cite{Dolan-Gavitt:2011uq} and VMST~\cite{Fu:2012fk}.
Existing systems, however, have a number of limitations. First, they
focus on retrieving kernel-level information. However, a great deal of
security-relevant information exists only at user-level, such as URLs
being visited by the browser, instant messages and emails sent by
desktop clients, and system and application log messages. Second,
they require that the desired information be accessible through some
public interface (a public API in the case of Virtuoso, and a userland
program or kernel module in the case of VMST). This means that some
security-relevant information may be inaccessible to such tools.
Finally, Payne et al.~\cite{payne:2008} argue that many security
applications need some form of \emph{active monitoring}; that is, they
need to be notified when certain system events occur. Current solutions
to the introspection problem provide no way of locating places in the
system where it would be useful to interpose.

In this paper, we attempt to address the limitations of past solutions
by examining a rich source of information about system and application
activity: memory accesses observed at runtime. Our key insight is that a
memory access made at a different points in a program can be treated as
a streams of information that are likely to be of the same type. For
example, when visiting a URL, a web browser must write to memory the URL
that is being visited, and it will generally do so at the same point in
the program; by intercepting memory accesses made at this program point
we can observe all URLs visited. These program points, which we call
\emph{tap points}, provide a natural place to interpose to extract
security-relevant information.

There are a several challenges that must be overcome to make use of tap
points. The first is the sheer amount of data that must be sifted
through. In ten minutes' worth of execution on a Windows 7 system, for
example, we observed a total of 18.9 \emph{million} unique tap points
which read and wrote a total of 32.8 gigabytes of data. To overcome this
challenge, we make use of techniques from information retrieval and
machine learning, described in Section~\ref{sec:technical}, to quickly
zero in on the tap points that read or write information relevant to
introspection.

Second, simply setting up an environment in which one can observe every
memory access made by the whole system (OS and applications) poses a
challenge. Whole-system emulators such as QEMU~\cite{Bellard:2005}
provide the necessary basis for such instrumentation, but intercepting
and analyzing every memory access online is not practical: the resulting
system is so slow that network connections time out and the guest OS may
think that programs have become unresponsive. To solve this problem, we
add \emph{record and replay} to QEMU, which allows executions to be
recorded with low overhead. Our heavyweight analyses are then run on
the replayed execution to analyze every memory access made
\emph{without} perturbing the system under inspection. We describe our
system, Tappan Zee Bridge (TZB),\footnote{So named because the
northbridge on Intel architectures traditionally carried data between
the CPU and RAM.} in detail in Section~\ref{sec:implementation}.

Finally, previous systems have required a significant amount of effort
to support new architectures. This problem has become more pressing in
recent years, as ARM-based devices such as smartphones have exploded in
popularity. Because TZB looks at memory accesses, rather than inspecting
binary code, it naturally supports a wide variety of architectures with
minimal effort. To demonstrate this, our evaluation includes the and ARM
architecture in addition to x86, and the techniques we describe easily
generalize to other architectures.

The remainder of this paper is structured as follows.
Section~\ref{sec:tapdef} defines precisely how we define a tap point. We
then explore that definition and its impact on the scope of our work and
the assumptions it rests on in Section~\ref{sec:scope}.
Section~\ref{sec:technical} describes techniques for finding tap points
of interest. We then discuss our system, Tappan Zee Bridge, which
implements these techniques, in Section~\ref{sec:implementation}. We
evaluate TZB in Section~\ref{sec:eval}, and show that it is capable of
finding tap points useful for introspection in a wide variety of
applications, operating systems, and architectures. Finally, we describe
the limitations of our approach in Section~\ref{sec:limitations},
related work in this area in Section~\ref{sec:relwork}, and offer
concluding remarks in Section~\ref{sec:conclusion}.
