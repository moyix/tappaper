\section{Technical Description}
\label{sec:technical}

To find useful \emph{tap points} in a system---places from which to
extract data for introspection---using Tappan Zee Bridge, one begins by
creating a recording that captures the desired OS or application
behavior. For example, if the end goal is to be notified each time a
user loads a new URL in Firefox, one would create a recording of Firefox
visiting several URLs. This recording is made by emulating the OS and
application inside of TZB, which can capture and record all sources of
non-determinism with low overhead, allowing for later deterministic
replay. Next, one can run one or more analyses that seek out the desired
information among all memory accesses seen during the execution.
Analyses in TZB take the form of plugins that are called on each memory
access made during a replayed execution and, at the end, write out a
report on the tap points analyzed. Finally, the tap points found should
be validated to ensure that they do, in fact, provide the desired
information. Such assurance can be gained either by examining the data
in the tap point in new executions, or by examining the code around the
tap point. This workflow is illustrated in Figure~\ref{fig:workflow}.

In this section, we give a technical description of Tappan Zee Bridge.
We describe three different ways of finding tap points according to a
standard epistemic classification scheme~\cite{Rumsfeld:2002}: searching
for ``known knowns''---tap points where the desired data and its format
is known; searching for ``known unknowns''---tap points where the kind
of data sought is known, but its precise format is not; and finally
``unknown unknowns''---tap points where the type and format of the data
sought are not known, and we are instead simply trying to find
``interesting'' tap points.

\subsection{Known Knowns}

The simplest case is finding data that one knows is likely to be read or
written by a tap point, and where the encoding of the data is easily
guessed. For example, to find a tap point that can be used to notify
the hypervisor whenever a URL is entered in a browser, one can visit a
known sequence of URLs, and then monitor all tap points, searching for
specific byte sequences that make up those URLs. The same holds for
other data whose representation when written to memory is predictable:
filenames, window titles, registry key names, and so on.

To realize this in TZB, we developed a plugin (\texttt{stringsearch})
that can quickly search for byte strings read or written by tap points
during a recorded execution. This can be accomplished with low overhead:
for each string sought and each tap point, we simply keep a counter that
tracks how many characters of the string have been matched so far, and
reset the counter as soon as a non-matching character is found. If this
counter ever equals the length of the string, then we can conclude that
the tap point has read or written the desired information.

\subsection{Known Unknowns}
\label{sec:technical:subsec:knownunk}

A second tap point application involves finding tap points for things
about which we have limited knowledge.

We can easily assemble corpora of exemplars to represent a semantic
class: English prose, kernel messages, or mail headers. These examples
need not come from tap points but can easily be collected directly from
interacting with the operating system itself. From such a corpus, we can
readily build a statistical model of the semantic class. Tap points
whose contents have high likelihood ratios for the semantic model with
respect to a null model are likely to come from the same semantic class
as the corpus. This kind of supervised learning permits us, for
instance, to train a model on examples of the contents of \texttt{dmesg}
under Linux and employ it to locate tap points that write analogous data
in other operating systems such as FreeBSD, Minix, and Haiku.

In TZB, we accomplish this by collecting bigram statistics for all tap
points seen in execution, as well as for the exemplar; the data seen at
each tap point is thus represented as a sparse vector with 65,536
elements (one for each possible pair of bytes). We can then sort the tap
points seen by taking the distance (according to some metric) from the
exemplar. For our metric, we have chosen to use Jensen-Shannon
divergence~\cite{Lin:2006fk}, which is a smoothed and symmetrized
version of the classic Kullback-Leibler
divergence~\cite{Kullback:1951uq} (also known as information gain). We
also examined the Euclidean and cosine distance metrics, but found their
performance to be consistently worse than Jensen-Shannon. Jensen-Shannon
divergence between two probability distributions $P$ and $Q$ is defined
as:

\[
JSD(P, Q) = H \left ( \frac{P+Q}{2} \right ) - \frac{H(P)+H(Q)}{2}
\]

\noindent where $H$ is Shannon entropy.

In addition to statistical comparison to a known exemplar, we can also
search for data of unknown format if we have access to some external
validator. This is the case, for example, when searching for tap points
that write encryption keys: although the exact key may not be known in
advance (ruling out the use of the \texttt{stringsearch} plugin), if we
have bit of encrypted data we can check whether a given byte string is a
valid decryption key.

In TZB, we currently implement this strategy for finding tap points that
write SSL/TLS master secrets, the 48-byte strings from which an
SSL/TLS-encrypted session's keys are derived. In the training phase, we
record the execution of a program that initiates an encrypted connection
to some server; we also record the encrypted packets sent by the client.
Then, using the \texttt{keyfind} plugin, we test each 48 bytes written
by each tap point and see whether the keys derived from it successfully
decrypt a sample packet sent by the client. If so, then we conclude that
the tap point can be used to decrypt SSL/TLS connections made by the
program under inspection. In Section~\ref{sec:eval:subsec:sslmal} we show
how this technique can be used to spy on connections made by the Sykipot
malware, without performing a (potentially detectable) man in the middle
attack.

\subsection{Unknown Unknowns}

The final strategy for finding useful tap points is also the least
focused. If there is no specific introspection quantity sought, one
might instead wish to find interesting tap points, for some suitable
definition of ``interesting.'' To support this scenario, TZB offers a
form of unsupervised learning---clustering---to group together tap
points that handle similar data. The idea is that one can then examine
exemplars from each cluster, rather than being forced to look through a
large number of tap points. Thus, our use of clustering functions as a
form of \emph{data triage}.

Our clustering is based on the venerable $k$-means
algorithm~\cite{Steinhaus:1956kx}, but using the Jensen-Shannon
divergence described in the previous section. As in the statistical
search case, we use bigram statistics for our feature vectors.
Initialization uses the KMeans++ algorithm~\cite{Arthur:2007ve}, which
helps guarantee that the initial cluster centers are widely separated.
We evaluate the performance of this clustering compared to an expert
labeling in Section~\ref{sec:eval:subsec:cluster}.
