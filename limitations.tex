\section{Limitations and Future Work}
\label{sec:limitations}

Although TZB is currently very useful for finding interception points
for \emph{active monitoring}, it is not currently usable in every
scenario where introspection is needed. Because the interception points
are triggered by executing code, they are only usable in \emph{online}
analysis. However, the need for introspection also arises in
\emph{post-mortem} analysis, specifically in forensic memory analysis.
Whereas previous solutions such as Virtuoso~\cite{Dolan-Gavitt:2011uq}
were able to operate equally well on memory images or live virtual
machines, TZB is only applicable to the live case. In future work, we
hope to combine Virtuoso-like techniques with TZB to produce offline
programs that can locate in memory the buffers on which TZB's tap points
operate.

Another limitation of TZB is its reliance on callstack information to
locate interposition points. In current systems, keeping track of an
arbitrary number of callers for each process is prohibitively expensive;
although stack walking is faster (since it only needs to be invoked when
the monitored code is executed), it is insecure, unreliable, and not
available on every architecture. We hope to examine how existing
solutions such as compiler modifications~\cite{stackshield,Chiueh:2001ys}
or dynamic binary translation~\cite{Sinnadurai:2008vn} can be used to
efficiently maintain the shadow stack information needed for TZB. We
also note that Intel's Haswell architecture, which as of this writing
has just been released, has hardware support for keeping track of calls
and returns~\cite{Haswell}; this would provide an excellent base on
which to build a low-overhead security system based on TZB tap points.

Code generated at runtime (i.e., JIT or injected code) may make
re-identification of a discovered tap point difficult or even
impossible. Given the rise of languages that depend on JIT runtimes,
better solutions are needed for this scenario, and the problem of how to
make use of tap points in JIT code should be explored further.

Finally, as seen in Section~\ref{sec:eval:subsec:cluster}, the
clustering results are promising, but not yet fully developed. We hope
to gain a better understanding of the data found in tap points and seek
out better features and models for clustering in future work.
