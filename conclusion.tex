\section{Conclusion}
\label{sec:conclusion}

In this paper we have presented TZB, a system that automatically locates
candidate memory accesses for active monitoring of applications or operating
systems. This is a task that previously required extensive reverse engineering
by domain experts. We have successfully used TZB to identify a broad range of
tap points, including ones to dynamically extract SSL keys, URLs typed into
browsers, and the names of files being opened. TZB is built atop the QEMU-based
PANDA platform as a set of plug-ins and its operation is operating system and
architecture agnostic, affording it impressive scope for application. This is a
powerful technique that has already transformed how the authors perform RE
tasks. By reframing a diffult RE task as a principled search through streaming
data provided by dynamic analysis, TZB allows manual effort to be refocused on
more critical and less automatable tasks like validation.
