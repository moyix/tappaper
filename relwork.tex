\section{Related Work}
\label{sec:relwork}

Although, to our knowledge, there is no existing work on mining the
contents of memory accesses for introspection, we drew inspiration from
a variety of sources. These can be roughly grouped into three
categories: work on automating virtual machine introspection, research
on automated reverse engineering, and efforts that examine memory access
patterns, typically through visualization. In this section, we describe
in more detail previous work in these three areas.

Virtual machine introspection has been targeted for automation by
several recent research efforts because of the \emph{semantic gap
problem}: security applications running outside the guest virtual
machine need to reconstruct high-level information from low-level data
sources, but doing so requires knowledge of internal data structures and
algorithms that is costly to acquire and maintain. To address this
problem, researchers have sought ways of bridging this gap
automatically. Virtuoso~\cite{Dolan-Gavitt:2011uq} uses dynamic traces
of in-guest programs to extract out-of-guest tools that compute the same
information. However, because it is based on dynamic analysis,
incomplete training may cause the generated programs to malfunction. Two
related approaches attempt to address this limitation: \emph{process
out-grafting}~\cite{Srinivasan:2011fk} moves monitored processes to the
security VM while redirecting their system calls to the guest VM,
allowing tools in the security VM to directly examine the process, while
VMST~\cite{Fu:2012fk} selectively redirects the memory accesses of tools
like \texttt{ps} and \texttt{netstat} from the security VM so that their
results are obtained from the guest VM. TZB extends these approaches by
finding points in applications and the OS at which to perform active
monitoring.

Based on the observation that memory accesses in dynamic execution can
reveal the structure of data in memory, several papers have proposed
methods for automatically deducing the structure of
protocols~\cite{Caballero:2007fk,Lin:2008uq,Cui:2007kx}, file
formats~\cite{Cui:2008ys,Lin:2008vn}, and in-memory data
structures~\cite{Lee:2011,Slowinska:2011ys,Lin:2010}. One particular
insight we have drawn from this body of work is the idea that the point
in a program at which a piece of data is accessed, along with its
calling context, can be used as a proxy for determining the type of the
data. TZB leverages this insight to separate out memory accesses into
streams of related data.

Finally, there has been some research on examining memory accesses made
by a single program or a whole system, typically using visualization.
Burzstein et al.~\cite{Bursztein:2011fk} found that by visualizing
the memory of online strategy games, they could identify the region of
memory used to decide how much of the in-game map was visible to the
player, which greatly reduces the work required to create a ``map hack''
and allow the player to see the entire map at once. Outside the academic
world, the ICU64 visual debugger~\cite{icu64} allows users to visualize
and modify the entire memory of a Commodore 64 system, enabling a
variety of cheats and enhancements to C64 games. Although TZB does not
use visualization, it shares with this previous work the understanding
that memory accesses can be a rich source of information about a running
program.
