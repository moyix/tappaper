\documentclass{sig-alternate}

\usepackage{url}
\newcommand{\fixme}{{\bf XXX\ }}

\begin{document}

%don't want date printed
\date{}

\title{Tappan Zee (North) Bridge: Mining Memory Accesses for Introspection}

%\numberofauthors{4}
%
%\author{
%\alignauthor
%Brendan Dolan-Gavitt\\
%    \affaddr{Georgia Institute of Technology}\\
%    \email{brendan@cc.gatech.edu}
%\alignauthor
%Josh Hodosh\\
%    \affaddr{MIT Lincoln Laboratory}\\
%    \email{josh.hodosh@ll.mit.edu}
%\and
%\alignauthor
%Tim Leek\\
%    \affaddr{MIT Lincoln Laboratory}\\
%    \email{tleek@ll.mit.edu}
%\alignauthor
%Wenke Lee\\
%    \affaddr{Georgia Institute of Technology}\\
%    \email{wenke@cc.gatech.edu}
%} % end author

\maketitle

% Use the following at camera-ready time to suppress page numbers.
% Comment it out when you first submit the paper for review.
%\thispagestyle{empty}

\begin{abstract}

The ability to introspect into the behavior of software at runtime is
crucial for many security-related tasks, such as virtual machine-based
intrusion detection and low-artifact malware analysis. Although some
progress has been made in this task by automatically creating programs
that can passively retrieve kernel-level information, two key challenges
remain. First, it is currently difficult to extract useful information
from user-level applications, such as web browsers. Second, discovering
points within the OS and applications to hook for \emph{active
monitoring} is still an entirely manual process. In this paper we
propose a set of techniques to mine the memory accesses made by an
operating system and its applications to locate useful places to deploy
active monitoring, which we call \emph{tap points}. We demonstrate the
efficacy of our techniques by finding tap points for four useful
introspection tasks such as finding SSL keys and monitoring web browser
activity on five different operating systems (Windows 7, Linux, FreeBSD,
Minix and Haiku) and two processor architectures (ARM and x86).

\end{abstract}

\section{Introduction}
\label{sec:introduction}

Many security applications have a need to inspect the internal workings
of software. Host-based intrusion systems, malware analyses, and
digital forensics all depend to some degree on being able to obtain
information about software that is by design undocumented and hidden
from public view. Thus, to operate correctly, security software is
typically built on \emph{reverse engineering}, the art and practice of
elucidating the undocumented principles on which software is built.

Unfortunately, reverse engineering is expensive, time consuming, and
requires a high degree of expertise. The problem is exacerbated by the
fact that, to protect against tampering, security applications are often
hosted in environments separated from the target being inspected, such
as a separate virtual machine. Because of this, their visibility into
the target is often limited to low-level features such as memory and CPU
state, and any higher-level information must be reconstructed based on
reverse engineered knowledge.

This problem, which we will refer to as the \emph{introspection
problem}, has been approached by a number of recent research efforts
such as Virtuoso~\cite{Dolan-Gavitt:2011uq} and VMST~\cite{Fu:2012fk}.
Existing systems, however, have a number of limitations. First, they
focus on retrieving kernel-level information. However, a great deal of
security-relevant information exists only at user-level, such as URLs
being visited by the browser, instant messages and emails sent by
desktop clients, and system and application log messages. Second,
they require that the desired information be accessible through some
public interface (a public API in the case of Virtuoso, and a userland
program or kernel module in the case of VMST). This means that some
security-relevant information, such as \fixme{example}, may be
inaccessible to such tools. Finally, Payne et al.~\cite{payne:2008}
argue that many security applications need some form of \emph{active
monitoring}; that is, they need to be notified when certain system
events occur. Current solutions to the introspection problem provide no
way of locating places in the system where it would be useful to
interpose.

In this paper, we attempt to address the limitations of past solutions
by examining a rich source of information about system and application
activity: memory accesses observed at runtime. Our key insight is that a
memory access made at a different points in a program \footnote{To deal
with bulk-copy functions such as \texttt{memcpy}, we require a
definition of a program point that is slightly stronger than just the
program counter.  We discuss this in more detail in Section
\ref{sec:technical}.} can be treated as a streams of information that
are likely to be of the same type. For example, when visiting a URL, a
web browser must write to memory the URL that is being visited, and it
will generally do so at the same point in the program; by intercepting
memory accesses made at this program point we can observe all URLs
visited. These program points, which we call \emph{tap points}, provide
a natural place to interpose to extract security-relevant information.

There are a several challenges that must be overcome to make use of tap
points. The first is the sheer amount of data that must be sifted
through. In ten minutes' worth of execution on a Windows 7 system, for
example, we observed a total of 18.9 \emph{million} unique tap points
which read and wrote a total of 32.8 gigabytes of data. To overcome this
challenge, we make use of techniques from information retrieval and
machine learning, described in Section~\ref{sec:technical}, to quickly
zero in on the tap points that read or write information relevant to
introspection.

Second, simply setting up an environment in which one can observe every
memory access made by the whole system (OS and applications) poses a
challenge. Whole-system emulators such as QEMU~\cite{Bellard:2005}
provide the necessary basis for such instrumentation, but intercepting
and analyzing every memory access online is not practical: the resulting
system is so slow that network connections time out and the guest OS may
think that programs have become unresponsive. To solve this problem, we
add \emph{record and replay} to QEMU, which allows executions to be
recorded with low overhead. Our heavyweight analyses are then run on
the replayed execution to analyze every memory access made
\emph{without} perturbing the system under inspection. We describe our
system, Tappan Zee Bridge (TZB),\footnote{Please accept our sincere
apologies for this pun.} in detail in
Section~\ref{sec:implementation}.

Finally, previous systems have required a significant amount of effort
to support new architectures. This problem has become more pressing in
recent years, as ARM-based devices such as smartphones have exploded in
popularity. Because TZB looks at memory accesses, rather than inspecting
binary code, it naturally supports a wide variety of architectures with
minimal effort. To demonstrate this, our evaluation includes the MIPS
and ARM architectures in addition to x86.

The remainder of this paper is structured as follows.
Section~\ref{sec:technical} describes techniques for finding tap points
of interest. We then discuss our system, Tappan Zee Bridge, which
implements these techniques, in Section~\ref{sec:implementation}. We
evaluate TZB in Section~\ref{sec:eval}, and show that it is capable of
finding tap points useful for introspection in a wide variety of
applications, operating systems, and architectures. Finally, we describe
the limitations of our approach in Section~\ref{sec:limitations},
related work in this area in Section~\ref{sec:relwork}, and offer
concluding remarks in Section~\ref{sec:conclusion}.


\section{Defining Tap Points}
\label{sec:tapdef}

\begin{figure}[t]
\begin{center}
\includegraphics[width=3.2in]{tappoint.pdf}
\end{center}
\caption{Three different ways of defining a tap point: (a) as a single
stream of information from the CPU to RAM ; (b) split up according to
program and location within program ; (c) split up according to program,
location within program, and calling context.}
\label{fig:tappoint}
\end{figure}

At the heart of our approach is an abstraction on top of memory accesses
made by the CPU, the \emph{tap point}. A tap point is a point in a
system at which we wish to capture a series of memory accesses for
introspection purposes; however, the exact definition of ``a point in a
system'' will make a great deal of difference in how effective our
approach can be.

A naive approach to defining tap points would be to simply group memory
accesses by the program counter that made them (e.g., EIP/RIP on x86 and
R15 on ARM). This approach fails in two common cases: first, memory
accesses made by bulk copy functions, such as \texttt{memcpy} and
\texttt{strcpy}, would all be grouped together, which would commingle
data from different parts of the program into the same tap point. In
addition, looking only at the program counter would conflate accesses
from different programs.

Instead, we define tap points as the triple
\[ (caller, program\_counter, address\_space) \]
Including the caller and the address space (the \texttt{CR3} register on
x86, and the \texttt{CP15 c2} register on ARM) separates out memory
accesses into streams that should, in general contain the same type of
data.\footnote{Making use of tap points defined this way in the real
world is slightly more difficult, since a program's address space will
differ and its code may be relocated by ASLR. These complications can be
overcome with a minor amount of engineering, however.} Figure
\ref{fig:tappoint} shows the effect of choosing various definitions of a
tap point when looking for the place where the browser writes the URL
entered by the user (``google.com''). At the coarsest granularity (a),
one can simply look at all writes from the CPU to RAM; however, the
desired information is buried among reams of irrelevant data.
Separating out tap points by program and program counter (b) is better,
but still combines uses of \texttt{strcpy} that contain different
information --- in this case, a filename and a URL. By including the
calling context (c), we can finally obtain a tap point that contains
just the desired information.

It is possible that some tap points may require deeper information about
the calling context (for example, if an application has its own wrapper
around \texttt{memcpy}), but in practice we have found that just one
level of calling context is usually sufficient. In addition, because TZB
uses a whole-system emulator that can watch every call and return, we
can obtain the call stack to an arbitrary depth for any tap point. This
makes it easy to add extra context for a given tap point, if it is found
that doing so separates out the desired information. Examples of tap
points that require more than one level of callstack information are
given in Sections \ref{sec:eval:subsec:ssl} and
\ref{sec:eval:subsec:file}.

Conversely, one might wonder whether this definition of a tap point may
split up data that should logically be kept together. To mitigate this
problem case, we introduce the idea of \emph{correlated tap points}: we
can run a pass over the recorded execution that notices when two tap
points write to adjacent locations in memory in a short period of time
(currently 5 memory accesses). The idea is that these tap points may be
more usefully considered jointly; for example, a single data structure
may have its fields set by successive writes. These writes would come
from different program counters, and hence would be split into different
tap points, but it may be more useful to examine the data structure as a
whole. By noticing this correlation we can analyze the data from the
combined tap point.


\section{Scope and Assumptions}
\label{sec:scope}

The goal of Tappan Zee Bridge is to find points at which to interpose
for active monitoring. More precisely, our goal is to speed the current
entirely manual process by which applications or operating systems are
reverse engineered in order to locate tap points for active monitoring.
It should be noted that we do not aim to surpass those manual efforts.
We have no automatic way, for instance, of knowing for certain if a tap
point will fail to output crucial data or, alternately, spew out
superfluous information under some future conditions. This is a separate
problem to which we see no ready solution. Static analysis of candidate
tap points or extensive testing are good stop-gaps, but nothing short of
fully understanding enormous binary code bases can really give complete
assurance that a tap point won't miss or cause false alarms in the
future. 

In this section, we explore how our definition of a tap point and our
focus on active monitoring shape the scope of our work.

First and most obviously, our focus on memory accesses necessarily
limits our scope to information that is read from or written to RAM at
some point. Although this is quite broad, there are notable exceptions.
For example, TRESOR~\cite{Muller:2011} performs AES encryption without
storing the key or encryption states in RAM by making clever use of the
x86 debug registers and the AES-NI instruction set. Aside from such
special cases, however, this assumption is not particularly limiting.

Second, our goal of finding tap points suitable for active monitoring
motivates a design that treats memory accesses at tap points as sources
of \emph{streaming} data. Our algorithms, therefore, typically work in a
streaming fashion as the system executes, remembering only a fixed
amount of state for each tap point. Although this is a natural fit for
active active monitoring, where events should be reported as soon as
possible, it makes handling data whose \emph{spatial} order in memory
differs from its \emph{temporal} order as it is accessed more difficult.

Finally, the use of calling context in the definition of a tap point
raises the question of how much context is necessary or useful. Our
current system uses only the most recent caller, but we have seen
both situations where this is not enough and where it is too much.
Overall, however, one level of calling context has proved to be a
reasonable choice for a wide variety of introspection tasks.

\begin{figure}[t]
    \begin{center}
        \includegraphics[width=3.2in]{figures/memaccess.pdf}
    \end{center}
    \caption{Patterns of memory access that we might wish to monitor
    using TZB.}
    \label{fig:memaccess}
\end{figure}

To better illustrate the boundaries of our technique, consider
Figure~\ref{fig:memaccess}, which plots the address of data written by
different tap points over time for four patterns of memory access. In
the bottom two quadrants, we have cases that are challenging, but
currently well-supported by TZB. In the bottom-left, a standard
\texttt{memcpy} implementation on x86 makes a copy in 4-byte chunks
using \texttt{rep movsd}, and then does a two-byte \texttt{movsw} to get
the remainder of the string. Because the access occurs across two
different instructions, TZB sees two different tap points. Our tap point
correlation mechanism correctly deduces that the accesses are related,
however, because they operate on adjacent ranges in a short span of
time.

The case shown in the bottom right quadrant would be tricky if we looked
only at memory access spatially and not temporally. Here, a utility
function writes data out to a serial port by making one-byte writes to a
memory-mapped I/O address.\footnote{Although not reported in this paper,
this case is one we actually encountered while experimenting with an
embedded firmware.} Because TZB sees these memory writes in
temporal order, ignoring the address, the data is seen normally and the
analyses we describe all operate correctly.

The upper quadrants show cases that are currently not handled by TZB. In
the upper left, \texttt{memmove} copies a buffer in reverse order when
the source and destination overlap. Thus, when viewed in temporal order,
a copy of a string like ``12345678'' would be seen by TZB as
``56781234''. This case is unlikely to be handled by TZB without a
significant redesign, as its view of memory accesses is inherently
streaming.

Finally, the upper right, which represents the case of \texttt{dmesg} on
Linux, is an example of the ``dilemma of context''. Although the
function, \texttt{do\_syslog}, that writes log data to memory  is called
from multiple places (creating multiple tap points), it writes to the
same contiguous buffer. Unlike the \texttt{memcpy} case, a significant
amount of time may pass before the next function calls
\texttt{do\_syslog}, and so our tap correlation, which only considers
memory accesses within a fixed time window, will not notice that the tap
points ought to be grouped together. We believe that this case could be
overcome with additional engineering work, but this is left to future
work.


\section{Search Strategies}
\label{sec:technical}

\begin{figure}
    \begin{center}
        \includegraphics[width=3.2in]{figures/tzbarch.pdf}
    \end{center}
    \caption{The workflow for using TZB to locate points at which to
    interpose for active monitoring.}
    \label{fig:workflow}
\end{figure}

To find useful \emph{tap points} in a system---places from which to
extract data for introspection---using Tappan Zee Bridge, one begins by
creating a recording that captures the desired OS or application
behavior. For example, if the end goal is to be notified each time a
user loads a new URL in Firefox, one would create a recording of Firefox
visiting several URLs. This recording is made by emulating the OS and
application inside of TZB, which can capture and record all sources of
non-determinism with low overhead, allowing for later deterministic
replay. Next, one can run one or more analyses that seek out the desired
information among all memory accesses seen during the execution.
Analyses in TZB take the form of PANDA plugins that are called on each
memory access made during a replayed execution and, at the end, write
out a report on the tap points analyzed. Finally, the tap points found
should be validated to ensure that they do, in fact, provide the desired
information. Such assurance can be gained either by examining the data
in the tap point in new executions, or by examining the code around the
tap point. This workflow is illustrated in Figure~\ref{fig:workflow}.

In this section, we describe three different ways of finding tap points
grouped according to a standard epistemic classification
scheme~\cite{Rumsfeld:2002}: searching for ``known knowns''---tap points
where the desired data and its format is known; searching for ``known
unknowns''---tap points where the kind of data sought is known, but its
precise format is not; and finally ``unknown unknowns''---tap points
where the type and format of the data sought are not known, and we are
instead simply trying to find ``interesting'' tap points.

\subsection{Known Knowns}

The simplest case is finding data that one knows is likely to be read or
written by a tap point, and where the encoding of the data is easily
guessed. For example, to find a tap point that can be used to notify
the hypervisor whenever a URL is entered in a browser, one can visit a
known sequence of URLs, and then monitor all tap points, searching for
specific byte sequences that make up those URLs. The same holds for
other data whose representation when written to memory is predictable:
filenames, window titles, registry key names, and so on. For this kind
of data, simple string searching is usually sufficient to zero in on the
few tap points that handle the data of interest, and in our experience
it is one of the most effective techniques for finding useful tap
points.

\subsection{Known Unknowns}
\label{sec:technical:subsec:knownunk}

A second tap point application involves finding tap points for things
about which we have limited knowledge.

We can easily assemble corpora of exemplars to represent a semantic
class: English prose, kernel messages, or mail headers. These examples
need not come from tap points but can easily be collected directly from
interacting with the operating system itself. From such a corpus, we can
readily build a statistical model of the semantic class. Tap points
whose contents have high likelihood ratios for the semantic model with
respect to a null model are likely to come from the same semantic class
as the corpus. This kind of supervised learning permits us, for
instance, to train a model on examples of the contents of \texttt{dmesg}
under Linux and employ it to locate tap points that write analogous data
in other operating systems such as FreeBSD, Minix, and Haiku.

In addition to statistical comparison to a known exemplar, we can also
search for data of unknown format if we have access to some external
validator. This is the case, for example, when searching for tap points
that write encryption keys: although the exact key may not be known in
advance (ruling out the use of string matching), if we have bit of
encrypted data we can check whether a given byte string is a valid
decryption key by trying to decrypt our sample data.

\subsection{Unknown Unknowns}

The final strategy for finding useful tap points is also the least
focused. If there is no specific introspection quantity sought, one
might instead wish to find interesting tap points, for some suitable
definition of ``interesting.'' To support this scenario, TZB offers a
form of unsupervised learning---clustering---to group together tap
points that handle similar data. The idea is that one can then examine
exemplars from each cluster, rather than being forced to look through a
large number of tap points. Thus, our use of clustering functions as a
form of \emph{data triage}.


\section{Implementation}
\label{sec:implementation}

In this section, we describe both the dynamic analysis platform employed to
build TZB, but also TZB-specific algorithmic and data-structure solutions.

\subsection{PANDA}
\label{sec:implementation:subsec:panda}

TZB makes extensive use of the Platform for Architecture-Neutral Dynamic
Analysis (PANDA), which was developed by the authors in collaboration
with Northeastern University. A brief description of PANDA follows.

PANDA is based upon version 1.0.1 of the QEMU machine
emulator~\cite{Bellard:2005}. QEMU is an excellent and common choice
for whole-system dynamic analysis for two main reasons. First,
performance is good (about 5x slowdown over native). Second, every basic
block of guest code is disassembled by the host in order to emulate,
which means that there are opportunities to interpose analyses at the
basic block or even instruction level, if desired. Disassembly and code
generation is fairly sophisticated in that QEMU lowers instructions to
an intermediate language (IL) in order to employ a single back-end code
generator, the Tiny Code Generator (TCG). This IL means dynamic
analyses can potentially be written once and re-used for all 14
architectures supported by QEMU. Further, this version of QEMU is modern
enough to be able to boot and run modern operating systems such as
Windows 7 (earlier versions of QEMU such as 0.9.1 cannot).

There are three main aspects to PANDA that make it very convenient for
building dynamic analyses. None are earth-shattering in
their novelty, but they compose to form a unique and particularly useful
system that deserves description (a more detailed technical report on
PANDA is forthcoming). First, PANDA provides a plug-in architecture that
readily permits writing guest analyses in C and C++. Plug-in code is
executed from a number of standard callback locations including
instructions, before and after basic blocks, hypercalls, system calls,
physical and virtual memory read and writes, etc. This is not unlike the
schemes employed in other whole-system dynamic analysis platforms such
as BitBlaze~\cite{Song:2008bitblaze} and S2E~\cite{Chipounov:2011s2e}.In
addition, plugins can export functionality that can then be used in
other plugins, allowing complex functionality to be built up from simple
components. For example, most plugins in TZB use the callstack plugin
(described below) to determine their calling context; this means that
the analysis plugins need not worry about the messy details of how to
track calls and returns across different architectures. From a software
engineering perspective, PANDA's plugin architecture allows the various
analyses supported by TZB to be cleanly separated from the main
emulator, which makes for a much more comprehensible and maintainable
codebase.

The second aspect of PANDA that makes it an excellent dynamic analysis
platform is nondeterministic record and replay (RR). In our formulation
of RR, we begin a recording by invoking QEMU's built-in snapshot
capability. Subsequently, we record all inputs to the CPU, including
INs, interrupts, and DMA. Recording imposes a small overhead (10-20\%)
but not enough to perturb execution. During replay, we revert to a
snapshot and proceed to pull CPU inputs from a log when required.
Unlike many other RR schemes, we do not record and replay device inputs,
which means we cannot ``go live'' at any point during replay. But we
can perform repeated replays of an entire operating system under
arbitrary instrumentation load without worrying about this perturbing
application or operating system operation. This capability is vital to
TZB: without record and replay, the heavyweight analyses we perform
would make the system unusably slow.

The final aspect of PANDA worth mentioning is its integration of LLVM.
QEMU lowers basic blocks of guest code to its own IL, which PANDA can,
additionally, re-render as basic blocks of LLVM coden via a module 
extracted from S2E. We will not discuss this capability any further 
here as it is not used by TZB.


\subsection{Callstack Monitoring}
\label{sec:implementation:subsec:callstack}

As explained in Section~\ref{sec:tapdef}, tap points need information
about the calling context. Keeping track of this information requires
some knowledge about the CPU architecture on which the OS is running,
and so we decided to encapsulate this task into a single plugin. TZB's
other analyses can then query the current call stack to arbitrary depth
by invoking \texttt{get\_callers} and not worry about the details
described in this section.

To track call stack information, the \texttt{callstack} plugin examines
each basic block as it is translated, looking for an
(architecture-specific) call instruction. For x86, we rely on the
diStorm3~\cite{distorm} disassembler, which supports both x86 and amd64
and abstracts away the many different forms the call instruction can
take. On ARM, the instruction set is relatively simple, and we just use
a poor man's disassembler based on matching bytes, looking for two
idioms that indicate a call: \texttt{bl} and \texttt{mov lr, pc}. If the
block includes a call instruction, then we push the return address onto
a shadow stack after each time that block executes.

Detecting the return from a function does not require any
architecture-specific code. Before the execution of every basic block,
we check whether the address we are about to execute is at the top of
the stack; if so, we pop it. We only need to check the starting address
of the basic block, because by definition a return terminates a basic
block, so the return address will always fall at the beginning of a
block.

An additional complication is that a single process may have multiple
threads, which implies multiple independent call stacks. To deal with
this, we implemented an optional feature in the call stack plugin that
monitors changes in the stack pointer, and creates a new shadow call
stack when the value of the stack pointer jumps by more than a fixed
threshold (currently 5000 bytes, a parameter we determined empirically).
This extra precision comes at a performance penalty, however
(approximately 2x slowdown over normal shadow stack tracking), and for
TZB we usually only need one level of caller information, so the results
given in this paper do not use this feature.

We note that these techniques may fail if traditional call/return
semantics are violated. For example, if a program emulated calls and
returns by manually pushing the return address and using a direct jump,
it would not be detected as a call. However, for non-malicious
compiler-generated code, we have found that the algorithm described here
works well.

\subsection{Fixed String Searching}
\label{sec:implementation:subsec:stringsearch}

Searching for fixed strings is one of the most effective tools for
finding useful tap points. Because we have to sift through many
gigabytes of data that pass through tap points during any given
execution, it is vital that string search be efficient in both time and
space.

To satisfy these constraints, we developed a string matching algorithm
that requires only one byte of memory per search string and per tap
point. This one-byte counter tracks, for a given tap point, how many
bytes of the search string have been matched by the data seen at the tap
point so far. Whenever a byte is read from or written to memory, we can
check what the next byte in the search string is using this position,
and compare it to the byte passing through the tap point. If it matches,
the counter is incremented; if it does not match, the counter is reset
to zero. When the counter equals the length of the search string, we
know that the search string has passed through the tap point, and we
report a match. Note that because the counter is only one byte, our
matcher only supports strings up to 256 bytes long; this cap could be
easily raised to 65,536 bytes by using a two-byte counter, at the cost
of doubling the memory requirements. Thus far, 256-byte strings have
been more than sufficient.

This effectively implements a very simple deterministic finite automaton
(DFA) matcher. Indeed, we believe that it should be possible to
efficiently implement a streaming basic regular expression matcher that
only an amount of memory logarithmic in the number of states needed to
represent the expression. We leave this generalization to future work,
however.

\subsection{Bigram Collection and Statistical Search}
\label{sec:implementation:subsec:bigram}

Collecting bigram statistics on data that passes through each tap point
is an efficient way to enable ``fuzzy'' search based based on some
training examples, as well as enabling clustering. Bigram collection is
done by maintaining, for each tap point, two pieces of information: (1)
the last byte that passed through by the tap point, so that we can see
bigrams that span a single memory access; (2) a histogram of all byte
pairs seen at the tap point. The latter of these must be maintained
sparsely: because our bigrams are based on bytes, a dense histogram
would require 65,536 integers' worth of storage per tap point. Given
that most of the executions examined in this paper contain upwards of
500,000 tap points, this would require more than 120GB of memory, which
is clearly infeasible (and wasteful, since most of those entries would
be zero).

Instead, we store the histogram sparsely, using a C++ Standard Template
Library \texttt{std::map<uint16\_t,int>}. This keeps memory usage down
without sacrificing any accuracy, but it does introduce some extra
complexity when processing the resulting histograms, as our search
software must support sparse vectors rather than simple arrays. Because
of this additional complexity, we opted to implement the search and
clustering algorithms ourselves, after some initial prototyping using
SciPy's \texttt{sklearn} toolkit.

Our statistical search tool is implemented in 246 lines of C++, and
computes the Jensen-Shannon divergence between a training histogram
(dense) and a set of sparse histograms. Our K-Means clustering tool is
481 lines of C++ code, and outputs a clustering of the sparse histograms
using Jensen-Shannon divergence as a distance metric.\footnote{The use
of this distance metric is justified theoretically because
Jensen-Shannon distance is a Bregman divergence~\cite{Banerjee:2005qf}
and empirically because our clustering typically converges after around
30 iterations.} Both tools are multithreaded, which greatly speeds up the
computation.


\section{Evaluation}
\label{sec:eval}

In this section, we evaluate the efficacy of our various tap point search
strategies, described in Section~\ref{sec:technical}, at finding tap
points useful for introspection. Our experiments are motivated by
real-world introspection applications, and so for each experiment we
describe a typical application for the tap points found. Each experiment
was also generally performed on a variety of different operating sytems,
applications, and architectures in order to evaluate TZB's ability to
handle a diverse range of introspection targets.

For the sake of readability, we have attempted to use symbolic names for
addresses wherever possible in the following results. It is hoped that
these will be more meaningful to the reader than the raw addresses, but
we emphasize that debug information is in no way required for TZB to
work.

\subsection{Known Knowns}

\subsubsection{URL Access}
\label{sec:eval:subsec:url}

\begin{table*}
    \centering
    \small
    \begin{tabular}{|l|l|l|}
        \hline
        Browser & Caller & PC \\
        \hline
        Deb Epiphany (arm) & \texttt{WebCore::KURL::KURL+0x30} & \texttt{WebCore::KURL::init+0x70} \\
        Deb Epiphany (amd64) & \texttt{webkit\_frame\_load\_uri+0xc3} & \texttt{WebCore::KURL::init+0x368} \\ 
        Win7 IE8 (x86) & \texttt{ieframe!CAddressEditBox::\_Execute+0xaa} & \texttt{ieframe!StringCchCopyW+0x50} \\
        Win7 Firefox (x86) & \texttt{xul!nsAutoString::nsAutoString+0x1a} & \texttt{xul!nsAString\_internal::Assign+0x1d} \\
        Win7 Chrome (x86) &  \texttt{msftedit!CTxtEdit::OnTxChar+0x105} & \texttt{msftedit!CTxtSelection::PutChar+0xb8} \\
        Win7 Opera (x86) &  \texttt{Opera.dll+0x2cf6c6} & \texttt{Opera.dll+0x142783} \\
        Haiku WebPositive (x86) & \texttt{BWebPage::LoadURL+0x3a} & \texttt{BMessage::AddString+0x26} \\
        \hline
    \end{tabular}
\caption{Tap points found that write the URL typed into the browser by
the user.}
\label{tbl:url}
\end{table*}

Monitoring visited URLs is likely to be useful for host-based intrusion
detection and prevention systems. For example, an IDS may wish to verify
that outgoing requests were initiated by a human rather than malware on
the users's machine, or match URLs visited against a blacklist of
malicious sites. This poses a challenge for existing introspection
solutions, as URL load notification is not generally exposed by a public
API, and the data resides in a user application (the browser).

To find URL tap points, we created training executions by visiting a set
of three URLs (Google, Facebook, and Bing) in the following operating
systems and browsers: Epiphany on Debian squeeze (armel and amd64);
Firefox 16.0.2, Opera 12.10, and Internet Explorer 8.0.7601.17514 on
Windows 7 SP1 (x86); and WebPositive r580 on Haiku (x86). We used the
\texttt{stringsearch} plugin to search for the ASCII and UTF-16
representations of the three URLs, and then validated each tap point
found to ensure that it wrote only the desired data. The results can be
seen in Table~\ref{tbl:url}.

\subsubsection{TLS/SSL Master Secrets}
\label{sec:eval:subsec:ssl}

\begin{table*}
    \centering
    \small
    \begin{tabular}{|l|l|l|l|}
        \hline
        Client & Caller & PC & Process \\
        \hline
        Deb OpenSSL (arm) & \texttt{tls1\_generate\_master\_secret+0x9c} & \texttt{tls1\_PRF+0x90} & openssl \\
        Deb OpenSSL (amd64) & \texttt{ssl3\_send\_client\_key\_exchange+0x437} & \texttt{tls1\_generate\_master\_secret+0x108} & openssl \\
        Deb Epiphany (arm) & \texttt{md\_write+0x74} & \texttt{md5\_write+0x68} & epiphany \\ 
        Deb Epiphany (amd64) & \texttt{md\_write+0x60} & \texttt{md5\_write+0x49} & epiphany \\ 
        Haiku WebPositive (x86) & \texttt{tls1\_generate\_master\_secret+0x65} & \texttt{tls1\_PRF+0x14b} & WebPositive \\
        Win7 Chrome (x86) & \texttt{chrome!NSC\_DeriveKey+0x1241} & \texttt{chrome!TLS\_PRF+0xa0} & chrome.exe \\
        Win7 IE8 (x86) & \texttt{ncrypt!Tls1ComputeMasterKey@32+0x57} & \texttt{ncrypt!PRF@40} & lsass.exe \\
        Win7 Firefox (x86) & \texttt{softokn3!NSC\_DeriveKey+0xe85} & \texttt{freebl3!TLS\_PRF+0xbb} & firefox.exe \\
        Win7 Opera (x86) & \texttt{Opera.dll+0x2eb06e} & \texttt{Opera.dll+0x50251} & opera.exe \\
        \hline
    \end{tabular}
\caption{Tap points found that write the SSL/TLS master secret for each
SSL/TLS connection.}
\label{tbl:ssl}
\end{table*}

Monitoring SSL/TLS-encrypted traffic is a classic problem for intrusion
detection systems. Currently, hypervisor- or network- based IDSes that
wish to analyze encrypted traffic must perform a man-in-the-middle
attack on the connection, presenting a false server certificate to the
client. Not only does this require the client to cooperate by trusting
certificates signed by the intrusion detection system, it also takes
control of the certificate verification process out of the hands of the
client---a dangerous step, given that many existing SSL/TLS interception
proxies have a history of certificate trust
vulnerabilities~\cite{JarmocBHEU2012}.

Instead of a man-in-the-middle attack, we can instead use TZB to find a
tap point that reads or writes the SSL/TLS master secret for each
encrypted connection, giving us a ``man-on-the-inside''. Because this
secret must be generated for each SSL/TLS connection, if we can find
such a tap point, it can then be provided to the IDS to decrypt and, if
necessary, modify the content of the SSL stream.

To find the location of these tap points, we ran a modified copy of
OpenSSL's \texttt{s\_server} utility that prints out the SSL/TLS master
key any time a connection is made. We then recorded executions in which
we visited the server with each of our tested SSL clients, and noted the
SSL/TLS master secret. Finally, we used \texttt{stringsearch} to search
for a tap point that wrote the master key, and verified that the tap
wrote exactly one master key per connection. For this test, we used:
OpenSSL s\_client 0.9.8 on Debian squeeze (armel), OpenSSL s\_client
0.9.8 and Epiphany 2.30.6 on Debian squeeze (amd64), and Firefox
16.0.2, Google Chrome 23.0.1271.64, Opera 12.10, and Internet Explorer
8.0.7601 on Windows 7 SP1 (x86). The results are shown in
Table~\ref{tbl:ssl}.

There is one particular point of interest to observe in these results.
In the case of Epiphany on Debian, we found that one level of callstack
information was \emph{not} sufficient---with only the immediate caller,
the tap point contains more data than just the SSL/TLS master secret.
This is because the version of Epiphany uses SSLv3 to make connections,
and the pseudo-random function (PRF) used in SSLv3 has the form
\[ MD5(SHA1(\ldots))\] The other implementations instead use TLSv1.0, where
the PRF has the form \[ MD5(\ldots) \oplus SHA1(\ldots) \] This final
XOR operation is done from a unique program point, so the tap point that
results from it contains only TLS master keys. This points to a
potential complication of using tap points for introspection: it is not
always clear in advance how many levels of call stack information will
be required.

We were successful in locating tap points for all SSL/TLS clients tested.
We note that uncovering similar information using traditional techniques
would have required significant expertise and reverse engineering of
both open source and proprietary software.

\subsubsection{File Access}
\label{sec:eval:subsec:file}

\begin{table*}
    \centering
    \small
    \begin{tabular}{|l|l|l|}
        \hline
        Target & Caller & PC \\
        \hline
        Debian (amd64) & \texttt{getname+0x13e} & \texttt{strncpy\_from\_user+0x52} \\ 
        Debian (arm) & \texttt{getname+0x88} & \texttt{\_\_strncpy\_from\_user+0x10} \\
        Haiku (x86) & \texttt{EntryCache::Lookup+0x27} & \texttt{hash\_hash\_string+0x1b} \\
        FreeBSD (x86) & \texttt{namei+0xd1} & \texttt{copyinstr+0x38} \\
        Windows 7 (x86) & \texttt{ObpCaptureObjectName+0xcb} & \texttt{memcpy+0x33} \\
        \hline
    \end{tabular}
\caption{Tap points found for file access on different operating systems.}
\label{tbl:file}
\end{table*}

Monitoring file accesses is a requirement for many host-based security
applications, including on-access anti-virus scanners. Thus, locating a
tap point at which system-wide file accesses can be observed is of
considerable importance. However, because previous approaches to the
introspection problem~\cite{Dolan-Gavitt:2011uq,Fu:2012fk} passively
retrieve information from the guest and are not event-driven, they
cannot be used in this scenario.

To find such a tap point, we created recordings in which we opened files
in various operating systems. Specifically, in each OS we created 100
files, each named after ten successive digits of $\pi$. The operating
systems chosen for this test were: Debian squeeze (amd64), Debian
squeeze (armel), Windows 7 SP1 32-bit, FreeBSD 9.0, and Haiku R1 Alpha
3 (all on x86). We then searched for tap points that wrote strings
matching the ASCII and UTF-16 encodings of the filenames using the
\texttt{stringsearch} analysis plugin. The UTF-16 encodings were
included because it was known that Windows 7 uses UTF-16 for strings
pervasively, allowing us to surmise that on Windows URLs would likely be
UTF-16 encoded. Finally, we looked at the tap points found by
\texttt{stringsearch}, and validated them by hand.

The results are shown in Table~\ref{tbl:file}. For most of the operating
systems we had no difficulty finding a tap point that contained the name
of each file as it was accessed. The one exception was Windows 7, where
the most promising tap point not only wrote file results, but also a
number of unrelated objects such as registry key names. As in the SSL
case, the root cause of this was insufficient calling context: in
Windows several different things fall under the umbrella of a ``named
object'', and these were all being captured at this tap point. We found
that four levels of calling context were sufficient to restrict the tap
point to just file accesses; the ``deepest'' caller was
\texttt{IopCreateFile} (which, despite its name, is used for both
opening existing files and creating new ones).

\subsection{Known Unknowns}
\subsubsection{SSL Malware}
\label{sec:eval:subsec:sslmal}

The need to snoop on SSL-encrypted connections arises in malware
analysis as well. Two features distinguish this case from that of
intercepting the traffic of benign SSL clients presented in the previous
section. First, the ability to decrypt the traffic without a man in the
middle is even more important: in contrast to benign clients, we cannot
assume that malware will accept certificates signed by our certificate
authority. Second, we cannot rely on having access to the server's
master secret, as the server is under the attacker's control. This means
that our previous strategy of using a simple string search for the
master secret will not work here.

Instead, we located the tap point in the SSL-enabled malware
using our \texttt{keyfind} plugin, which performs trial decryption on a
packet sent by the malware using each possible 48-byte sequence written
to memory as a key and verifies whether the Message Authentication Code
is valid. Although this is much slower than a string match, it is the
only available option, since the key is not known in advance.

To test the plugin, we obtained a copy of a version of the Sykipot
trojan released around October 31st, 2012~\cite{sandymal} (MD5:
\texttt{34a1010846c0502f490f17b66fb05a12}). We then created a recording
in which we executed the malware; simultaneously, we captured network
traffic using \texttt{tcpdump}. We noted that the malware made several
encrypted connections to \url{https://www.hi-techsolutions.org/}, and
provided one of the encrypted packets from these connections as input to
the \texttt{keyfind} plugin. The plugin found the same tap point as the
Windows 7 IE8 experiment described earlier, indicating
that both the malware and IE8 likely use the same underlying system
mechanism to make SSL connections. The key found was able to decrypt the
connections contained in the packet dump.\footnote{The malware also has
a second layer of encryption, which is custom and not based on SSL; we
did not attempt to decrypt this second layer.}

\subsubsection{Finding \large \texttt{dmesg}}
\label{sec:eval:subsec:dmesg}

\begin{table*}
    \centering
    \small
    \begin{tabular}{|l|l|l|l|l|}
        \hline
        OS & Caller & PC & Kernel? & Rank \\
        \hline
        FreeBSD (x86) & \texttt{msglogstr+0x28} & \texttt{msgbuf\_addstr+0x19a} & Yes & 1 \\
        Haiku (x86) & \texttt{ring\_buffer\_peek+0x59} & \texttt{memcpy\_generic+0x14} & Yes & 1 \\
        Debian (arm) & N/A & \texttt{do\_syslog+0x18c} & Yes & 4 \\
        Debian (amd64) & N/A & \texttt{do\_syslog+0x163} & Yes & 4 \\ 
        Minix (x86) & \texttt{0x190005ee} & \texttt{0x190009d4} & No & 8 \\
        Windows 7 (x86) & Not Found & Not Found & ? & ? \\
        \hline
    \end{tabular}
\caption{Tap points that write the system log (\texttt{dmesg}) on
several UNIX-like operating systems. All tap points were located in the
kernel, except for Minix, which is a microkernel. We were unable to find
a tap point analogous to \texttt{dmesg} in Windows.}
\label{tbl:dmesg}
\end{table*}

System logs are an invaluable resource, both for security and system
administration. In an introspection-based security system, for example,
one might want to find a tap point that contains the system's logs so
that they can be stored securely outside the guest virtual machine.
However, because the format of system logs is particular to each OS, we
need some mechanism that can find tap points that write data that
``looks like'' a log based on an exemplar. The statistical search
described in Section~\ref{sec:technical:subsec:knownunk} is a good fit
for this task: by training on the output of \texttt{dmesg} on one
OS, we can find \texttt{dmesg}-like tap points on other systems.

To locate these system log tap points, we first created a training
exemplar by running the \texttt{dmesg} command on a Debian sid (amd64)
host and computing the bigram probabilities for the output. We then
created recordings in which we booted five operating systems (Debian
squeeze (armel), Debian squeeze (amd64), Minix R3-2.0 (i386), FreeBSD
9.0-RELEASE (i386), and Haiku R1 Alpha 3 (i386)), and computed the same
bigram statistics. We then sorted the tap points seen in each operating
system boot according to their Jensen-Shannon distance from the training
distribution, and manually examined data written by the tap point for
each of the top 30 results in each operating system.
Table~\ref{tbl:dmesg} shows, for each operating system tested, the tap
point that we determined to be the system log, and its rank in the
search results.

We can see that in all cases the correct result is in the top 10. There
are two additional features of Table~\ref{tbl:dmesg} that bear
mentioning. First, the reader will note that the two Debian systems have
a caller of ``N/A''. This is because the memory writes that make up
\texttt{dmesg} are done in \texttt{do\_syslog}, which is called from
multiple functions. In these cases, including the caller splits up
information that is semantically the same. We detected this case by
noticing that several of the top-ranked results in the Linux experiments
had the same program counter, and that they appeared to contain
different sections of the same log. Second, the tap point found for
Haiku was also incomplete---some lines were truncated. By using our tap
point correlation plugin, we determined that we were missing a second
tap point that was correlated with the main one; the two together formed
the write portions of a \texttt{memcpy} of the log messages. Once this
second tap point was included, we could see all the log messages
produced by Haiku.

We also attempted to find an analogous log message tap point on Windows
7, but were not successful. This is a result of the way Windows logging
works: rather than logging string-based messages, applications and
system services create a manifest declaring possible log events, and
then refer to them by a generated numeric code. Human-readable messages
are not stored, and instead are generated when the user views the log.
This means that there is no tap point that will contain log messages of
the type used in our \texttt{dmesg} training, and the methods described
in this paper are largely inapplicable unless a training example for the
binary format can be found. However, because the event log query API
is public~\cite{evtquery}, existing tools such as
Virtuoso~\cite{Dolan-Gavitt:2011uq} might be a better fit for this use
case.

Anecdotally, the ability to uncover a tap point that writes the kernel
logs has also been useful for diagnosing problems when adding support
for new platforms to QEMU. For an unrelated research task, we attempted
to boot the Raspberry Pi~\cite{raspberrypi} kernel inside QEMU, but
found that it hung without displaying any output early on in the boot
process. By locating the \texttt{dmesg} tap point, we discovered that
the last log message printed was ``Calibrating delay loop...''; based on
this we determined that the guest was hung waiting for a timer interrupt
that was not yet implemented in QEMU.

\subsection{Unknown Unknowns: Clustering}
\label{sec:eval:subsec:cluster}

To test the effectiveness of clustering tap points based on bigram
statistics and Jensen-Shannon distance, we carried out an experiment
that compared the clusters generated algorithmically to a set of labels
generated by two of the co-authors manually examining the data. We
created six recordings representing different workloads on two operating
systems (Windows 7 and FreeBSD 9.0). From FreeBSD, we took recordings of
boot, shutdown, running applications (\texttt{ps}, \texttt{cat},
\texttt{ls}, \texttt{top}, and \texttt{vi}), and a one-minute recording
of the system sitting idle, for a total of four recordings. On Windows
we created two recordings: running applications (\texttt{cmd.exe},
\texttt{dir}, the Task Manager, Notepad), and one minute of the system
sitting idle.\footnote{Although we would have preferred to include
Windows boot and shutdown recordings, at the time our replay system had
a bug (now fixed) that prevented these recordings from being replayed.}

Next, we sampled a subset of the tap points found in each recording.
Given that the vast majority of tap points do not write interesting
information, we opted not to sample uniformly from the all tap points
found. Instead, we performed an initial $k$-means clustering with $k =
100$, and then picked out tap points at various distances from each
cluster center. We chose the tap point at $\sigma$
standard deviations from the center, for $\sigma \in \left\{0, 0.5,
0.75, 1.0, 1.25, 1.5, 1.75, 2.0 \right\}$ for a total of 2,926
samples\footnote{The alert reader will note that this is smaller than
the 4,800 samples one would expect from taking 8 samples from 100
clusters in each of 6 recordings. This is because some clusters did not
have very high variance, and so in many cases there were fewer than 8
samples at the required distance from the center.}. Finally, we dumped
the data from each of the sampled tap points, blinded them by assigning
each a unique id, and then provided the data files to our two labelers.
Each labeler independently assigned labels to each of the samples using
the labels described in Table~\ref{tbl:clustlabels}, and the two
labelers then worked together to reconcile their labels.  

Finally, we ran a $k$-means clustering with $k = 10$; $10$ was chosen
because it was a round number reasonably close to the number of labels our
human evaluators gave to the data. We then used the Adjusted Rand
Index~\cite{Hubert:1985zr} to score the quality of our clustering
relative to our hand-labeled examples. The Adjusted Rand Index for a
clustering ranges from -1 to 1; clusterings which are independent of the
hand labeling will receive a score that is negative or close to zero. As
can be seen in Table~\ref{tbl:clustqual}, our clustering did not match 
up very well on the hand-labeled samples.  Note, however, that labeling
criteria were selected without knowledge of the sizes of categories or
whether or not the distance metric would effectively discriminate, so it
is perhaps unsurprising that the correspondence is poor.

There is some hope, however.  
Regardless of the apparently poor clustering performance with respect to 
hand-labeling, we decided to determine if the clusters from our 100-mean
clustering of FreeBSD's
boot process contained new and interesting data and if finding that data
would be facilitated by them.  
First, we determined to which of the clusters data from the
FreeBSD's \texttt{dmesg} and filename tap points (found in Sections
\ref{sec:eval:subsec:dmesg} and \ref{sec:eval:subsec:file}) was assigned.  
We were heartened to learn that these two text-like tap points 
had been sent to the same cluster.
We proceeded to explore this cluster of approximately 5000 tap points, 
and found that, indeed, the vast majority of the tap points contained 
readable text of some sort.  
Further, in the course of about thirty minutes of spelunking around this
cluster, we found not only kernel messages and filenames but a 
stone soup of shell scripts, process listings kernel configuration, 
GraphViz data, and so on.
A selection of these tap point contents is provided in 
Appendix~\ref{app:tap-point-contents}.
We did not exhaustively examine this cluster, but plan to do so soon,
as it appears to contain much of interest for active monitoring.
If clustering has focused us on one out of 100 clusters, this is 
potentially a big savings.

\begin{table}
    \centering
    \small
    \begin{tabular}{|l|l|r|}
        \hline
        Abbrv. & Description & Count \\
        \hline
        bp  &  binary pattern &  2318  \\
        rd  &  repeated dword &  400  \\
        mz  &  mostly zero &  141  \\
        rq  &  repeated quadword &  19  \\
        fnu  &  filenames unicode &  8  \\
        woa  &  words ascii &  8  \\
        wou  &  words unicode &  7  \\
        inu  &  integers unicode &  6  \\
        bu  &  binary uniform &  5  \\
        ura  &  URLs ascii &  5  \\
        rs  &  repeated short &  4  \\
        fna  &  filenames ascii &  2  \\
        rb  &  repeated byte &  2  \\
        vr  &  very redundant &  1  \\
        \hline
    \end{tabular}
\caption{Labels given to the sampled tap points by human evaluators,
along with the number of times each occurred.}
\label{tbl:clustlabels}
\end{table}

\begin{table}
    \centering
    \small
    \begin{tabular}{|l|r|}
        \hline
        Recording & ARI \\
        \hline
        FreeBSD Apps & 0.018 \\
        FreeBSD Boot & 0.048 \\
        FreeBSD Idle & 0.021 \\
        FreeBSD Shutdown & 0.074 \\
        Win7 Apps & 0.029 \\
        Win7 Idle & -0.003 \\
        \hline
    \end{tabular}
\caption{Quality of clustering as measured by the Adjusted Rand Index,
which ranges from -1 to 1, with 1 being a clustering that perfectly
matches the hand-labeled examples.}
\label{tbl:clustqual}
\end{table}

\subsection{Accuracy}
\label{sec:eval:subsec:accuracy}

Leaving aside the clustering results for the moment, the analyses
implemented in TZB are extremely effective at helping to identify
interposition points for active monitoring. In the evaluations based on
string searching, we found that the number of tap points we had to look
at manually was at most 262 (URLs under IE8) and in the best case we
only had to examine two tap points (for SSL keys under Firefox, Opera,
Haiku, and OpenSSL on ARM). The number of tap points that need to be
examined is related to how widely the data is propagated in the system
and how common the string being searched for is; thus, it is natural
that URLs visited in the browser would appear in many tap points,
whereas the SSL/TLS master key would not. Qualitatively speaking, we
found that once the candidate tap points had been selected by
\texttt{stringsearch} for a given execution, it took at most an hour to
find one that sufficed for the task at hand.

For the \texttt{dmesg} evaluation, we also examined the quality of the
results found for each operating system using the standard ``Precision
at 10'' metric, which is just the number of results found in the top 10
that were actually relevant to the query. In this case, this is simply
the fraction of results in the top 10 that appeared to contain the
system log (even if it was incomplete). Based on this metric, the
precision of our retrieval was between 20\% (on Minix) and 100\% (on
Haiku). This means that if one looked at all of the top 10 entries, it
is guaranteed that one would find the correct tap point.


\section{Limitations and Future Work}
\label{sec:limitations}

Not applicable to memory analysis \\

\noindent Call stack information may be unavailable / costly \\

Remember to cite the notion of a shadow stack, and see if we can find
something about the security of stack-walking.

\noindent
Attacks

\begin{itemize}
\item Split up data reads/writes into noncontiguous chunks and process separately -- obscure semantics
\item Randomize tap locations using something like JIT
\end{itemize}


\section{Related Work}

We use QEMU~\cite{Bellard:2005}, and this stuff could be used in a system like Lares~\cite{payne:2008}.


\section{Conclusion}
\label{sec:conclusion}

In this paper we have presented TZB, a system that automatically locates
candidate memory accesses for active monitoring of applications or operating
systems. This is a task that previously required extensive reverse engineering
by domain experts. We have successfully used TZB to identify a broad range of
tap points, including ones to dynamically extract SSL keys, URLs typed into
browsers, and the names of files being opened. TZB is built atop the QEMU-based
PANDA platform as a set of plug-ins and its operation is operating system and
architecture agnostic, affording it impressive scope for application. This is a
powerful technique that has already transformed how the authors perform RE
tasks. By reframing a diffult RE task as a principled search through streaming
data provided by dynamic analysis, TZB allows manual effort to be refocused on
more critical and less automatable tasks like validation.


\bibliographystyle{abbrv}
\bibliography{biblio}

\appendix
\label{app:tap-point-contents}
\section{Sample Tap Point Contents}


Here, we reproduce a selection of tap points from the same cluster as
dmesg and filename tap points.

\small

\begin{verbatim}
/etc/rc.d/ipfw/etc/rc.d/NETWORKING/etc/rc.d/netwait/etc/rc
.d/mountcritremote/etc/rc.d/devfs/etc/rc.d/ipmon/etc/rc.d/
mdconfig2/etc/rc.d/newsyslog
\end{verbatim}

\begin{verbatim}
r=/sNsnWs/fuiebu/ r=ceremdsecd_t_co_artpachg=tSooadSebaabf
/faa_N_=_peOfA=fA=feTr=tul.n=_eo/.b_Yt_vtectvifat=a=-sd_Ee
Ofu=u_0y:nF:tRseeeeEfciOtmdtuinlrlrrlpp/nppfpcepinl=l=.llN
lNlgllpl_.4l_l_2/l_l_22lileldlylo- 21laltlat=rrrsbgrskgni/
\end{verbatim}

\begin{verbatim}
russian|Russian Users Accounts:  :charset=KOI8-R:        :
lang=ru_RU.KOI8-R:     :       :passwd_format=md5:     :co
pyright=/etc/COPYRIGHT:      :welcome=/etc/motd:     :sete
nv=MAIL=/var/mail/$,BLOCKSIZE=K,FTP_PASSIVE_MODE=YES:     
\end{verbatim}

\begin{verbatim}
nss_compat.so.1dhclientShared object ``nss_compat.so.1'' n
ot found, required by ``dhclient''nss_nis.so.1dhclientShar
ed object ``nss_nis.so.1'' not found, required by ``dhclie
nt''nss_files.so.1dhclientShared object ``nss_files.so.1''
\end{verbatim}

\begin{verbatim}
digraph geom {
z0xc1d8de00 [shape=box,label=''PART\nada0\nr#2''];
z0xc1f4f640 [label=''r1w0e0''];
z0xc1f4f640 -> z0xc1e9eb00;
\end{verbatim}

\begin{verbatim}
/sbin/in/bin/sh/bin/stt/sbin/sysctl/bin/ps/sbin/sysctl/sbi
n/rcorde/bin/cat/sbin/md/sbin/sysctl/sbin/sysctl/bin/ken/s
bin/dumpon/bin/ln/bin/ps/sbin/sysctl/sbin/sysctl/sbin/sysc
tl/sbin/sysctl/bin/ps/bin/dd/sbin/sysctl/bin/dat/bin/df/sb
\end{verbatim}

\begin{verbatim}
/boot/kernel/kernel00000000-0000-0000-0000-000000000000000
00000-0000-0000-0000-00000000000000000000-0000-0000-0000-0
00000000000993c915d-3e9f-11e2-a557-525400123456993c915d-3e
9f-11e2-a557-525400123456/boot/kernel/kernel/boot/kernel/k
...
modulesoptions       CONFIG_AUTOGENERATED
ident   GENERIC
machine i386
cpu     I686_CPU
cpu     I586_CPU
cpu     I486_CPU
\end{verbatim}

\begin{verbatim}
<mesh>
  <class id=''0xc10362c0''>
    <name>FD</name>
  </class>
  <class id=''0xc1009a80''>
\end{verbatim}

\begin{verbatim}
#!/bin/sh
#
# $FreeBSD: release/9.0.0/etc/rc.d/newsyslog 197947 2009-1
0-10 22:17:03Z dougb $
...
set_rcvar()
{
        case $# in
        0)
                echo ${name}_enable
                ;;
        1)
\end{verbatim}


\begin{figure}[t!]
    \begin{center}
        \includegraphics[width=2in]{figures/freebsd_graphviz.pdf}
    \end{center}
    \caption{Detail from rendering of Graphviz file captured from a FreeBSD
boot tap point, apparently depicting disk geometry}
    \label{fig:graphviz}
\end{figure}






\end{document}
